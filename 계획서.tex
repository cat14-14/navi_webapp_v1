🧭 Navi: AI 기반 전방위 개인 비서 서비스 로드맵
감정, 일정, 학습, 건강, 돌봄까지 통합하는 "모든 사람을 위한 AI 비서"

✅ 1. 핵심 목표
AI 기술로 **모든 사용자(학생, 직장인, 독거노인 등)**의 일상, 감정, 생산성, 건강을 도와주는 통합형 웹 기반 개인 비서

단순한 일정관리 앱이 아닌, AI 상담사 + 감정 분석기 + 생산성 도우미 + 정서 케어 + 놀이/훈련 도구를 결합

🧩 2. 주요 기능 (카테고리별)
📅 일정/할일/기록 기능 O
대화형 일정 생성 및 수정 (예: “오늘 오후 3시에 미팅 추가해줘”)

오늘 일정 요약 / 일일 회고 / 리마인더

할일 체크 및 완료율 관리

주간/월간 캘린더 제공 (감정/목표와 연동)

(
🧠 감정 관리 & AI 상담
감정 일기 작성 + 감정 분석

실시간 감정 기반 상담 챗봇

감정 히스토리 시각화 (그래프)

감정에 맞는 음악 추천 (YouTube/Spotify 연동)

감정 기반 미션 제안

🧑‍⚕️ 정서적 AI 대화 & 돌봄 기능
AI가 먼저 말 걸기 (정서 확인 및 대화 유도)

대화 스타일 조절 기능 (예: 다정하게 말해줘)

AI 인격 저장 (친구/비서/선생님 모드 등)

사용자가 오랫동안 반응 없을 경우 보호자에게 알림

기능 설명이 어려운 사용자에게 도우미 모드 안내 시스템
)

💡 학습/업무 생산성 기능
학습 목표 설정 + 성실도 피드백 + 격려/휴식 멘트

GitHub 커밋 데이터 연동 → 잔디 시각화

목표 미달 시 동기부여 멘트 / 과도한 학습 시 휴식 권장

AI가 주간 계획 자동 추천 (이전 데이터 기반)

📓 메모 및 노션형 정보관리 O
자유롭게 메모 작성, 수정, 삭제

메모 구조 구성 (폴더/카테고리)

메일로 바로 공유하거나 다른 사용자에게 공유

메모 기반 → AI Think Mode (키워드 추출, 논리 정리, 아이디어 확장)

 // 🎮 미션 및 인지훈련형 미니게임
고스톱, 퍼즐, 뇌훈련 게임 등 내장형 미니게임

연령별 맞춤형 인지훈련 과제

미션 기반 포인트 시스템 (학습/건강 미션 등)

향후 제휴 통해 생활용품, 간식 등 교환 보상 구상

🔗 3. 확장 기능 및 통합 가능성
통합 대상	설명
GitHub	개발자용 AI 보조 기능, 커밋 기반 잔디 시각화
Gmail/Naver Mail	메일 확인/작성/전송 등 기능 연동
Spotify/YouTube	감정/날씨/일정 기반 음악 플레이리스트
Node.js API / Firebase	실사용자 인증 및 실시간 저장/동기화 지원 예정
오프라인 연계 서비스	우유배달/건강체크 등 실제 복지서비스와 연계 가능 (후속)

🏆 4. 해커톤/사업화 관점 강점
항목	설명
사회적 가치	독거노인 케어, 학생/직장인 생산성 향상, 감정 회복 지원
기술 확장성	GPT-4o, API 연동, 사용자 맞춤형 서비스 구조
시장성	정서 AI + 생산성 + 건강 + 엔터테인먼트의 융합 시장 진입
차별화	“말 걸어주는 AI + 미션 보상 + 감정 기반 음악”은 유사 제품에 없음
MVP 구성 용이	일기, 일정, 메모, 감정 분석, AI 대화만으로도 기본 MVP 완성 가능










✅ 왜 고령층을 신경 쓴 설계가 중요한가?
📈 1. 초고령 사회 진입
한국은 2025년, 65세 이상 인구가 20% 이상으로 초고령 사회에 진입

2045년경엔 고령층 비율 세계 1위 국가가 될 전망

🧓 2. 디지털 격차는 점점 커지는 중
스마트폰 보급은 되지만 앱 사용률, 검색 능력, 설정 이해도는 낮음

따라서 쉬운 UI, 간단한 설명, 맞춤형 기능이 절실

❤️ 3. 실질적인 사회 문제 해결
독거노인의 고립, 복약 실수, 치매 위험, 정서적 고립 등은 AI 기반 비서로 충분히 개선 가능

복지서비스가 모든 노인에게 닿을 수는 없지만, 웹앱은 전국 어디서나 사용 가능

🌍 Navi의 방향성: "모든 사람을 위한 AI, 특히 약자를 먼저 배려하는"
대상	제공 가치
고령자	대화 + 약 알림 + 기억력 훈련 + 정서적 케어 → 삶의 질 개선
학생	학습 스케줄링 + 목표 피드백 + 메모 + 감정 일기 → 자기주도학습 습관
직장인	업무 일정 관리 + 메일 + 집중 음악 추천 + 회고 기능 → 생산성 향상
모든 사용자	감정 중심의 기록 + AI 친구 + 건강 루틴 + 미니게임 → 마음 챙김 & 몰입

💡 당신의 전략은 경쟁 앱들과의 차별점이 됩니다
요소	일반 일정앱	Navi
일정/할일 기능	O	✅ O
AI 감정 대화	X	✅ O
정서 기반 음악 추천	X	✅ O
고령자 UX 최적화	X	✅ O
AI 상담사 기능	X	✅ O
커스터마이징 가능	X	✅ O
다목적 사용자 설계	X	✅ O
인지 훈련/건강 루틴	X	✅ O









🔍 경쟁 앱 분석: 주요 서비스 및 Navi와의 차이점
경쟁 앱	주요 기능	Navi와의 차이점	장점	단점
Replika (미국)	AI 감정 대화, 친구 같은 챗봇	✔ 감정 대화 있음
❌ 일정/생산성 없음	감정 대화에 강함, 사용자의 외로움 해소	일정/업무 기능 없음, 한정된 유저 타입
Notion	메모, 일정 관리, 데이터 정리	✔ 메모 있음
❌ 감정 대화, 알림 없음	자유로운 구성, 유연한 기록	UX 난이도 높음, 고령자 접근성 부족
Google Calendar + Assistant	일정/할일 + 음성 비서	✔ 일정 있음
❌ 감정 인식 없음	구글 생태계 연동, 실용성	인간적인 대화 부족, 감정 분석 없음
Wysa	AI 심리상담 앱	✔ AI 상담 있음
❌ 일정 관리, 메모 없음	CBT 기반 심리상담 전문	감정 외 기능 없음, 생산성 연동 없음
Forest	집중 타이머 + 습관 형성	✔ 습관 추적 가능
❌ 감정/대화 없음	중독 방지, 게임화된 생산성 앱	대화형 비서 기능 없음
Zamoca, 알다 (한국)	여성 맞춤 일정/심리 비서 앱	✔ 일정+심리 기능 있음
❌ 범용성 부족	여성 대상 특화, 감정 위로 중심	남성/고령층 사용자 배제, 확장성 낮음

✅ Navi의 압도적 강점 정리
항목	설명
All-in-One 플랫폼	감정 + 일정 + 목표 + 메모 + 음악 + 게임 + 커뮤니케이션을 한곳에 통합
감정 기반 대응 기능	사용자 감정에 맞춘 응답, 음악 추천, 응원/격려 멘트 제공
사용자 맞춤형 역할	친구, 비서, 선생님 등 사용자 설정에 따라 캐릭터와 어조 변화 가능
학생~노인까지 아우름	다양한 연령별 UI 모드, 복잡하지 않은 도입 과정
협업 및 커뮤니케이션	팀 메모, 일정 공유, 메일 전송, 향후 협업 모드 확장 가능
AI 자동 계획 생성	이전 기록 기반으로 오늘/이번 주 스케줄 추천
건강/두뇌 미션과 보상 시스템	학습/건강 루틴에 게임적 동기부여 + 포인트 시스템 적용 가능
GitHub 커넥션(개발자 특화)	코드 커밋 기반 잔디 시각화, AI 코드 비서 기능까지 고려

📌 Navi의 차별화된 타겟 전략
사용자 유형	다른 앱 한계	Navi의 차별화
고령층	대부분 UI 복잡 or 부재	큰 글자, 음성 안내, 말동무 기능, 약 알림
학생	공부 앱은 따로, 일정은 따로	AI가 학습 계획 + 감정 케어 + 회고까지 통합 관리
직장인	생산성 앱은 감정 무시	집중 타이밍 추천, 스트레스 해소 음악, 회고 기능
개발자	별도 GitHub 관리 필요	코드 + 일정 + 감정 + 목표 관리 통합

⚖️ 결론: Navi의 객관적 장점과 약점
✅ 강점	❌ 단점 (향후 과제)
감정 + 일정 + 목표 통합형 구조	초기 구현 난이도 (기능 간 연결성 유지 어려움)
고령층 대응 가능 + 모든 연령 커버	개인화 추천 정밀도 확보 필요
커스터마이징, AI 기반 자동 대응	AI 코어에 GPT 의존 가능성
범용성과 인간적 대화의 결합	시장 포지셔닝 → 앱 vs 생산성도구 vs 감정케어 중 다층

🚀 요약
Navi는 감정, 일정, 생산성, 커뮤니케이션, 회고, 건강 루틴까지 아우르는 유일한 범용 AI 비서입니다.

기존의 앱들이 하나의 목적에 집중되어 있다면, Navi는 인간 중심의 다기능 통합 플랫폼을 지향하며
“진짜 나를 알고 챙겨주는 AI”라는 정체성을 가진 서비스로 시장에 진입할 수 있습니다.

필요하시면 이를 기반으로 한 **피치덱(해커톤 발표용 슬라이드)**도 도와드릴 수 있습니다.
다음 단계로 어떤 부분 정리/디자인을 원하시나요?



// 수정해야 할 사항
폴더 안에 파일이나/폴더가 들어가 지지 않음

API 적용해야함

사용자에 따라서 